%%%%%%%%%%%%%%%%%%%%%%%%%%%%%%%%%%%%%%%%%
% Journal Article
% LaTeX Template
% Version 1.4 (15/5/16)
%
% This template has been downloaded from:
% http://www.LaTeXTemplates.com
%
% Original author:
% Frits Wenneker (http://www.howtotex.com) with extensive modifications by
% Vel (vel@LaTeXTemplates.com)
%
% License:
% CC BY-NC-SA 3.0 (http://creativecommons.org/licenses/by-nc-sa/3.0/)
%
%%%%%%%%%%%%%%%%%%%%%%%%%%%%%%%%%%%%%%%%%

%----------------------------------------------------------------------------------------
%	PACKAGES AND OTHER DOCUMENT CONFIGURATIONS
%----------------------------------------------------------------------------------------

\documentclass[twoside,twocolumn]{article}
\usepackage[english, russian]{babel}
\usepackage{graphicx}
\usepackage[utf8]{inputenc}
\usepackage{blindtext} % Package to generate dummy text throughout this template 

\usepackage[sc]{mathpazo} % Use the Palatino font
\usepackage[T1]{fontenc} % Use 8-bit encoding that has 256 glyphs
\linespread{1.05} % Line spacing - Palatino needs more space between lines
\usepackage{microtype} % Slightly tweak font spacing for aesthetics

\usepackage[english]{babel} % Language hyphenation and typographical rules

\usepackage[hmarginratio=1:1,top=32mm,columnsep=20pt]{geometry} % Document margins
\usepackage[hang, small,labelfont=bf,up,textfont=it,up]{caption} % Custom captions under/above floats in tables or figures
\usepackage{booktabs} % Horizontal rules in tables

\usepackage{lettrine} % The lettrine is the first enlarged letter at the beginning of the text

\usepackage{enumitem} % Customized lists
\setlist[itemize]{noitemsep} % Make itemize lists more compact

\usepackage{abstract} % Allows abstract customization
\renewcommand{\abstractnamefont}{\normalfont\bfseries} % Set the "Abstract" text to bold
\renewcommand{\abstracttextfont}{\normalfont\small\itshape} % Set the abstract itself to small italic text

\usepackage{titlesec} % Allows customization of titles
\renewcommand\thesection{\Roman{section}} % Roman numerals for the sections
\renewcommand\thesubsection{\roman{subsection}} % roman numerals for subsections
\titleformat{\section}[block]{\large\scshape\centering}{\thesection.}{1em}{} % Change the look of the section titles
\titleformat{\subsection}[block]{\large}{\thesubsection.}{1em}{} % Change the look of the section titles

\usepackage{fancyhdr} % Headers and footers
\pagestyle{fancy} % All pages have headers and footers
\fancyhead{} % Blank out the default header
\fancyfoot{} % Blank out the default footer
\fancyhead[C]{The Kernel Trick $\bullet$ Декабрь 2019 $\bullet$ МФТИ} % Custom header text
\fancyfoot[RO,LE]{\thepage} % Custom footer text

\usepackage{titling} % Customizing the title section

\usepackage{hyperref} % For hyperlinks in the PDF
\usepackage{amssymb,amsmath,amsthm}

\theoremstyle{plain}
\newtheorem{theorem}{Теорема}
\newtheorem{lemma}{Лемма}
\newtheorem{proposition}{Утверждение}
\newtheorem{corollary}{Следствие}
\theoremstyle{definition}
\newtheorem{definition}{Определение}
\newtheorem{notation}{Обозначение}
\newtheorem{example}{Пример}
\DeclareMathOperator{\sign}{sign}

%----------------------------------------------------------------------------------------
%	TITLE SECTION
%----------------------------------------------------------------------------------------

\setlength{\droptitle}{-4\baselineskip} % Move the title up

\pretitle{\begin{center}\Huge\bfseries} % Article title formatting
\posttitle{\end{center}} % Article title closing formatting
\title{Компьютерное моделирование идеального газа, распределение Максвелла, флуктуации.} % Article title
\author{%
\textsc{Н. В. Павличенко, А. С. Подкидышев} \\[1ex] % Your name
\normalsize Московский физико-технический институт \\ % Your institution
\normalsize \href{mailto:pavlichenko.nv@phystech.edu}{pavlichenko.nv@phystech.edu} 
\href{mailto:pavlichenko.nv@phystech.edu}{podkidishev.as@phystech.edu}% Your email address
%\and % Uncomment if 2 authors are required, duplicate these 4 lines if more
%\textsc{Jane Smith}\thanks{Corresponding author} \\[1ex] % Second author's name
%\normalsize University of Utah \\ % Second author's institution
%\normalsize \href{mailto:jane@smith.com}{jane@smith.com} % Second author's email address
}
\date{\today} % Leave empty to omit a date
\renewcommand{\maketitlehookd}{%
\begin{abstract}
\noindent В данной статье расматривается ...
\end{abstract}
}

%----------------------------------------------------------------------------------------

\begin{document}

% Print the title
\maketitle

%----------------------------------------------------------------------------------------
%	ARTICLE CONTENTS
%----------------------------------------------------------------------------------------

\section{Введение}

\lettrine[nindent=0em,lines=3]{Л} {инейные методы} использовались с самого зарождения таких наук как статистика и машинное обучение.
 Они действительно хороши с теоретической точки зрения: для них доказано много теорем, найдены
довереительные интервалы для отклика в различных моделях и, более того, существует аналитическое решение для нахождения
оптимальных параметров алгоритма. Однако они имеют существенные проблемы: они очень плохо работают при нелинейных
зависимостях в данных. Собственно, хотелось бы иметь метод, который позволял бы использовать все достоинства линейных моделей
и, при этом, который позволял бы приближать и нелинейные зависимости определенных видов. Идея ядерных методов заключается в
том, что пространство признаков, в котором зависимость нелинейная, можно отобразить в другое пространство, в котором она уже
будет линейной. Это пространство называется \textbf{спрямляющим}. При этом на самом деле, достаточно знать только как выражается скалярное
произведение в новом пространстве, что мы и будем называть функцией ядра. Далее рассмотрим теоретический аспект подробнее.

%------------------------------------------------
\section{Задача линейной классификации}
Мы будем рассматривать задачу классификации объектов исходных данных. Каждый объект будет отнесён в класс 1 или 2 в зависимости от своих свойств, что очевидно мотивировано в задачах машинного обучения. Более конкретно, для заданного набора $\{x\}_{i = 1}^n \subset X$ мы хотим построить классификатор 
\[a: X \xrightarrow{} \{0, 1\}.\]
Понятно, что для реальных задач оправдано только $X = \mathbb{R}^n$, то есть классифицировать мы будем численные объекты с некоторым множеством признаков.

\section{Метод опорных векторов}


\section{Заключение}
В данной работе мы рассмотрели задачи классификации и регресси в машинном обучении. Рассмотрели метод опорных векторов как их возможное решение, поставили задачу
оптимизации и предложили алгоритм ее решения для обучения метода. Результаты показали, что метод применим для получения хорошего качества как на синтетических
данных, так и на реальных. Кроме того, идеи перевода признаков в спрямляющее пространство применимы не только для линейных алгоритмов, но и для более сложных
моделей.
%------------------------------------------------


%----------------------------------------------------------------------------------------
%	REFERENCE LIST
%----------------------------------------------------------------------------------------

\begin{thebibliography}{99} % Bibliography - this is intentionally simple in this template

\bibitem[Shawe-Taylor, J., Cristianini, N., 2004]{Shawe-Taylor:2004dg}
Shawe-Taylor, J., Cristianini, N. (2004).
\newblock Kernel Methods for Pattern Analysis.
\newblock {\em Cambridge University Press}

\bibitem[S. Boyd, L. Vandenberghe, 2009]{Shawe-Taylor:2009dg}
S. Boyd, L. Vandenberghe (2009).
\newblock Convex Optimization.
\newblock {\em Cambridge University Press}
 
\end{thebibliography}

%----------------------------------------------------------------------------------------

\end{document}
